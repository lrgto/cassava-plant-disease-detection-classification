
\textbf{Report Link:} \href{https://git.cs.kent.ac.uk/jl742/comp8260-group-project/-/tree/jl742-2}{https://git.cs.kent.ac.uk/jl742/comp8260-group-project/-/tree/jl742-2}

\section{Introduction}
\label{Introduction Section}

Artificial intelligence is used in various industrial and research fields to assist humanity in data analysis, medical care and recovery and more. To further enhance and explore the implementation of artificial intelligent systems detecting a disease within the cassava plant. In the project a collection of image of the plants will vary from healthy, slightly infected, fully infected and dead, "This group will endeavour to improve an existing classifier on a current data set. This groups goals are to experiment with the PCA, Decision Trees, Random Forests and Neural Networks on a data set to classify images displaying different stages of the ‘Cassava’ disease on plants. The data set [1] selected consists of thousands of images, we initial view this as an opportunity to compare the run time and accuracy of the machine learning script on different number of images supplied to the script, one could logically state that the more images analysed then the longer the run time and more accurate it will be to distinguish the identifier. \\

However what interests this group it how much longer will the run time be and how much will the accuracy change, once this analysis is complete, we can explore using different filters, change the script and change the image format to shorten the run time and improve accuracy. Upon initial review of the requirements, the data set is 12gb (Mostly copies of images in different formats) but only consists of 3.14gb of usable images, the Cassava data set [1] and script will be written in python which all members have and will be linked to commit to GitLab regularly. No other significant requirements are visible at this stage. The group chosen this data set as it's feasible that it can be manipulated and rewritten, we all are comfortable and believe the project goals are achievable with our current understanding. \\

The initial plan is to divide the workload whilst all committing to the project on GitLab. The report, research, presentation and analysis will be given to each member as their primary contribution to the project. Each group member will explore different number of images supplied to the final script to which will be analysed later via a graph displaying run time, accuracy against number of supplied images. The initial research is on-going currently as we begin to write the ML script, to which we will run a proof of concept and run 5k, 10k, 20k images and compare the run time, accuracy. In the second sprint we refine the Python script repeat the proof of concept which will give us a good comparison to how we have improved the Python script." \cite{Plan}
